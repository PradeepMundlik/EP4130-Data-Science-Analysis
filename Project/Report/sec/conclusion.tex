\section{Conclusion}
The Bayesian inferential framework coupled with the Metropolis-Hastings algorithm has been shown to be an effective approach for the analysis of complex statistical models, especially in scenarios burdened with high dimensionality. In this study, we successfully applied these methodologies to the problem of estimating vehicle stopping distances as a function of speed. By carefully selecting priors and constructing a likelihood function reflective of the assumed normal distribution of errors, we have utilized the Metropolis-Hastings algorithm to generate an ensemble of parameter samples. These samples allowed us to draw inferences about the slope, intercept, and variability of the regression line governing the relationship between speed and stopping distance. The convergence of the Markov chains from multiple initial values demonstrates the robustness of our approach and provides a solid foundation for the reliability of MCMC methods in statistical inference. The results underscore the utility of Bayesian analysis and MCMC in solving complex problems where traditional methods may falter due to the curse of dimensionality.
